\documentclass{article}
\usepackage{graphicx} % Required for inserting images
\usepackage{xcolor}

\title{Shifumi de Nash}
\author{Pierre Nunn et Léo Laffeach}

\begin{document}

\maketitle

\section{Objectifs pédagogiques}

Le but de cette activité est de faire comprendre aux élèves les notions d'équilibre de Nash et d'Optimum de Pareto en utilisant le jeu pierre-feuille-ciseaux ou shifumi.

\section{Définitions}

Optimum de Pareto : 

Un optimum de Pareto est une stratégie pour laquelle il n'y a pas de moyen d'augmenter les gains d'un joueur sans diminuer les gains d'un autre. 

C'est-à-dire, si on note les stratégies $s_1,..., s_n$, les gains du joueur i $g_i(s_1, ..., s_n)$ et $(s*_i)_{1\leq i \leq n}$ l'optimum de Pareto, on a que s'il existe un joueur i et des stratégies $(s'_i)_{1\leq i \leq n}$ telles que $g_i(s'_1, ..., s'_n) > g_i(s*_1, ..., s*_n)$ alors, il existe un joueur j tel que $g_j(s'_1, ..., s'_n) < g_j(s*_1, ..., s*_n)$.

Il peut y avoir plusieurs optimums de Pareto.
\\
\\
Optimum social :

Un optimum social est un optimum de Pareto particulier, pour lequel la somme des gains de tous les joueurs est optimale.

Ce n'est pas forcément un équilibre de Nash. (Dilemme du prisonnier)
\\
\\
Équilibre de Nash : 

Un équilibre de Nash est une stratégie qui maximise le gain de tous les joueurs.

C'est-à-dire que pour tout joueur i, $g_i(s*_1, ..., s_i, ..., s*_n) \leq g_i(s*_1, ..., s*_i, ..., s*_n)$

Il n'en existe pas toujours un.

\section{Activité}

Le principe de l'activité est que les élèves jouent des parties en un nombre fixé $n$ de manches de pierre-feuille-ciseaux. Avant qu'ils ne commencent une partie, un coup (pierre, feuille ou ciseaux) est annoncé à l'avance (ça peut-être le même coup pour toutes les parties ou ça peut se décider avec un dé à chaque fois). Les élèves gagneront des points selon le fait qu'ils aient joué ce qui est annoncé ou selon leur résultat face à ce que l'autre élève a joué.

On dit aux élèves que le but du jeu est d'avoir le plus de points possibles à la fin des $n$ manches.Certains voudront probablement essayer de battre leur adversaire en ayant plus de points que lui mais ce n'est pas une stratégie optimale pour gagner des points. Lors d'une remise en commun on comparera les stratégies de chaque paires de joueurs et on constatera (normalement) que ce sont les joueurs qui ont collaboré plutôt que de chercher à battre l'adversaire qui ont obtenu le plus de points et que dans ce jeu les joueurs ont un intérêt à s'entraider. (On peut amener à ce moment là les notions d'équilibre de Nash et d'optimum de Pareto).

\subsection{Règles}
\begin{itemize}
    \item Victoire : +2 points
    \item Égalité : +1 points
    \item Défaite : +0 points
    \item Annoncé : +1 points
\end{itemize}

\subsection{Annonce extérieure}
\begin{tabular}{c|c|c|c}
    J1/J2 & faible & annoncé & fort \\ \hline
    faible & 1/1 & \textcolor{red}{0/3} & 2/0  \\ \hline
    annoncé & \textcolor{red}{3/0} & \textcolor{purple}{2/2} & 1/2 \\ \hline
    fort & 0/2 & 2/1 & 1/1 
\end{tabular}

\begin{itemize}
    \item Bleu : Équilibre de Nash
    \item Rouge : Optimum de Pareto
    \item Violet : Les deux
\end{itemize}

\subsection{Annonce des joueurs (estimation de partie)}

Cet extension laisse les joueurs annoncer simultanément un coup. Elle n'est pas forcément intéressante en pratique : elle est difficile et ne permet pas de voir clairement la notion d'équilibre de Nash (sauf l'équilibre de Nash probabiliste mais quitte à en parler, autant le faire sur le jeu de Pierre-feuille-ciseaux normal sans annonce).

\begin{table}[h]
    \centering
    \begin{tabular}{cc||c|c|c||c|c|c||c|c|c|}
        J1/J2  & J1&   & Pierre &   &   & Feuille &   &   & Ciseaux &   \\ 
          J2   &   & P &   F    & C & P &    F    & C & P &    F    & C \\ \hline \hline
               & P &2/2&  2/1   &0/3&1/2&   3/1   &0/3&1/2&   2/1   &1/3\\ \hline
        Pierre & F &1/2&  1/1   &2/0&0/2&   2/1   &2/0&0/2&   1/1   &3/0\\ \hline
               & C &3/0&  0/2   &1/1&2/0&   1/2   &1/1&2/0&   0/2   &2/1\\ \hline \hline
               & P &2/1&  2/0   &0/2&1/1&   3/0   &0/2&1/1&   2/0   &1/2\\ \hline
        Feuille& F &1/3&  1/2   &2/1&0/3&   2/2   &2/1&0/3&   1/2   &3/1\\ \hline
               & C &3/0&  0/2   &1/1&2/0&   1/2   &1/1&2/0&   0/2   &2/1\\ \hline \hline
               & P &2/1&  2/0   &0/2&1/1&   3/0   &0/2&1/1&   2/0   &1/2\\ \hline
        Ciseaux& F &1/2&  1/1   &2/0&0/2&   2/1   &2/0&0/2&   1/1   &3/0\\ \hline
               & C &3/1&  0/3   &1/2&2/1&   1/3   &1/2&2/1&   0/3   &2/2\\ \hline
    \end{tabular}
    \caption{Score selon l'annoncé par rapport au joué pour chaque joueur}
\end{table}

\section{Variantes}

\subsection{Dilemme du prisonnier}

\begin{itemize}
    \item Jouer l'annoncé : Donner 2 points à l'autre
    \item Ne pas le jouer : Gagner un point
\end{itemize}

\begin{tabular}{c|c|c|c}
    J1/J2 & annoncé & autre  \\ \hline
    annoncé & \textcolor{red}{2/2} & \textcolor{red}{0/3}  \\ \hline
    autre & \textcolor{red}{3/0} & \textcolor{blue}{1/1} \\ \hline 
\end{tabular}

\subsection{Chicken}
Pénaliser le cas d'égalité dans l'annonce.
Si seulement un des joueurs change, qu'il soit gagnant ou perdant, les joueurs gagnent des points. Mais si les deux changent, les deux perdent des points. L'intérêt reste d'avoir le plus de points sans se concerter.

\begin{tabular}{c|c|c|c}
    J1/J2 & annoncé & autre  \\ \hline
    annoncé & {0/0} & \textcolor{red}{4/1}  \\ \hline
    autre & \textcolor{red}{1/4} & \textcolor{blue}{1/1} \\ \hline 
\end{tabular}

\section{Exemples informatiques}

\begin{itemize}
    \item TCP
    \item accessibilité à internet (wifi, 3G, ADSL...)
    \item équité entre les participants
\end{itemize}

\end{document}
